\section{Выводы}
В ходе выполнения лабораторной работы я освежил в памяти курс математической статистики: гистограмму, корелляцию и корреляционную матрицу 
для наборов данных. Так же я изучил библиотеку Pandas, она оказалась очень удобной для анализа данных.

Набор данных оказался не самым лучшим, с такими распределениями признаков получить высокую точность у линейных моделей может быть проблематично.

Был проанализирован набор данных Water quality \cite{kaggle}, результаты получились закономерные: безопасность воды равномерно скореллированна с другими признаками, но нашлись интересные зависимости: количество бактерий и количество вирусов, содержание хрома и хлорамина, или перхлората и серебра. Наверное, это можно описать химическими реакциями между веществами.
\pagebreak
